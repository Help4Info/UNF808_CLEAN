% ===================================================================
% SECTION REQUÊTES DE DÉTECTION
% ===================================================================

\section{Requêtes de Détection et Threat Hunting}
\label{sec:detection_queries}

\subsection{Requêtes OpenSearch Dashboards Query Language (DQL)}

Wazuh Dashboard utilise OpenSearch Dashboards Query Language (DQL), similaire à KQL, pour effectuer des recherches avancées dans les données indexées par Wazuh-Indexer. Cinq requêtes opérationnelles sont présentées ci-dessous pour chaque technique de mouvement latéral d'APT41.

\begin{figure}[H]
    \centering
    \includegraphics[width=\textwidth]{figures/Kestrel_console.png}
    \caption{Dashboard Kestrel}
    \label{fig:wazuh-dashboard-production}
\end{figure}

\subsubsection{DQL-1 : Détection Connexions RDP}

Cette requête identifie toutes les connexions RDP réussies (Event ID 4624 avec LogonType 10) dans l'index \texttt{wazuh-alerts-*}.

\begin{Verbatim}[frame=single, numbers=left, numbersep=5pt, fontsize=\footnotesize]
data.win.eventdata.logonType: "10" AND 
data.win.system.eventID: "4624" AND 
rule.id: 110001
\end{Verbatim}

Cette requête peut être affinée pour détecter spécifiquement les connexions administrateur :

\begin{Verbatim}[frame=single, numbers=left, numbersep=5pt, fontsize=\footnotesize]
rule.id: 110002 AND 
data.win.eventdata.targetUserName: (*admin* OR *adm_*)
\end{Verbatim}

Pour détecter les connexions RDP multiples depuis une même IP :

\begin{Verbatim}[frame=single, numbers=left, numbersep=5pt, fontsize=\footnotesize]
rule.id: 110003 AND 
rule.description: "Connexions MULTIPLES"
\end{Verbatim}

\subsubsection{DQL-2 : Détection Accès Partages Admin SMB}

Cette requête détecte les accès aux partages administratifs Windows (C\$, ADMIN\$, IPC\$) via nos règles Wazuh personnalisées.

\begin{Verbatim}[frame=single, numbers=left, numbersep=5pt, fontsize=\footnotesize]
rule.id: (110010 OR 110011) AND 
data.win.eventdata.shareName: (*C$ OR *ADMIN$ OR *IPC$)
\end{Verbatim}

Pour identifier spécifiquement les accès critiques C\$ et ADMIN\$ :

\begin{Verbatim}[frame=single, numbers=left, numbersep=5pt, fontsize=\footnotesize]
rule.id: 110011 AND 
data.win.eventdata.shareName: (*C$ OR *ADMIN$)
\end{Verbatim}

Pour détecter l'utilisation de PsExec :

\begin{Verbatim}[frame=single, numbers=left, numbersep=5pt, fontsize=\footnotesize]
rule.id: 110014 AND 
data.win.eventdata.objectName: (*PSEXESVC* OR *paexec* OR *remcom*)
\end{Verbatim}

\subsubsection{DQL-3 : Détection Exécution via WMI}

Cette requête identifie les processus lancés via WMI en détectant nos règles spécifiques.

\begin{Verbatim}[frame=single, numbers=left, numbersep=5pt, fontsize=\footnotesize]
rule.id: (110020 OR 110021) AND 
data.win.eventdata.parentImage: *WmiPrvSE.exe*
\end{Verbatim}

Pour détecter spécifiquement PowerShell via WMI (critique) :

\begin{Verbatim}[frame=single, numbers=left, numbersep=5pt, fontsize=\footnotesize]
rule.id: 110021 AND 
data.win.eventdata.image: (*powershell.exe* OR *pwsh.exe*)
\end{Verbatim}

Pour détecter les WMI Event Consumers (persistance) :

\begin{Verbatim}[frame=single, numbers=left, numbersep=5pt, fontsize=\footnotesize]
rule.id: 110024 AND 
rule.mitre.id: "T1546.003"
\end{Verbatim}

\subsubsection{DQL-4 : Détection Pass-the-Hash}

Cette requête détecte les authentifications NTLM suspectes indicatrices de Pass-the-Hash.

\begin{Verbatim}[frame=single, numbers=left, numbersep=5pt, fontsize=\footnotesize]
rule.id: (110030 OR 110033) AND 
data.win.eventdata.authenticationPackageName: "NTLM" AND 
data.win.eventdata.logonType: "3"
\end{Verbatim}

Pour détecter les anomalies WorkstationName (indicateur clé de PtH) :

\begin{Verbatim}[frame=single, numbers=left, numbersep=5pt, fontsize=\footnotesize]
rule.id: 110033 AND 
data.win.eventdata.workstationName: ("-" OR "WORKSTATION")
\end{Verbatim}

Pour détecter le credential dumping (accès à LSASS) :

\begin{Verbatim}[frame=single, numbers=left, numbersep=5pt, fontsize=\footnotesize]
rule.id: 110034 AND 
data.win.eventdata.targetImage: *lsass.exe*
\end{Verbatim}

\subsubsection{DQL-5 : Détection Pass-the-Ticket et Kerberos Anomalies}

Cette requête détecte les TGT avec chiffrement RC4 (downgrade attack) ou sans pré-authentification (Golden Ticket).

\begin{Verbatim}[frame=single, numbers=left, numbersep=5pt, fontsize=\footnotesize]
rule.id: (110041 OR 110042) AND 
data.win.eventdata.ticketEncryptionType: "0x17"
\end{Verbatim}

Pour détecter spécifiquement les Golden Tickets (sans pré-auth) :

\begin{Verbatim}[frame=single, numbers=left, numbersep=5pt, fontsize=\footnotesize]
rule.id: 110042 AND 
data.win.eventdata.preAuthType: "0"
\end{Verbatim}

Pour détecter le Kerberoasting (demandes multiples de Service Tickets) :

\begin{Verbatim}[frame=single, numbers=left, numbersep=5pt, fontsize=\footnotesize]
rule.id: 110044 AND 
rule.description: "Kerberoasting"
\end{Verbatim}

\subsubsection{DQL-6 : Détection Attaques Combinées}

Pour détecter les corrélations multi-techniques :

\begin{Verbatim}[frame=single, numbers=left, numbersep=5pt, fontsize=\footnotesize]
rule.id: (110050 OR 110051 OR 110052 OR 110053 OR 110054 OR 110055) AND 
rule.level: (14 OR 15)
\end{Verbatim}

Pour détecter spécifiquement le mouvement latéral massif :

\begin{Verbatim}[frame=single, numbers=left, numbersep=5pt, fontsize=\footnotesize]
rule.id: 110055 AND 
rule.description: "MOUVEMENT LATERAL MASSIF"
\end{Verbatim}

\subsection{Requêtes Kestrel Threat Hunting}

Kestrel est un langage déclaratif de threat hunting permettant d'effectuer des analyses sophistiquées avec corrélations temporelles et enrichissement threat intelligence. Pour l'utiliser avec Wazuh 4.11, nous devons configurer STIX-Shifter pour accéder aux données de Wazuh-Indexer.

\subsubsection{Installation et Configuration Kestrel}

Kestrel est installé sur le serveur Wazuh Manager (192.168.1.51) avec STIX-Shifter pour accéder aux données de Wazuh-Indexer.

\begin{Verbatim}[frame=single, numbers=left, numbersep=5pt, fontsize=\small]
# Installation dans environnement virtuel
python3 -m venv /opt/kestrel-venv
source /opt/kestrel-venv/bin/activate

# Installer Kestrel + STIX-Shifter
pip install kestrel-jupyter
pip install stixshifter
pip install stixshifter-modules-elastic_ecs

# Lancer Jupyter Notebook
jupyter notebook --ip=0.0.0.0 --port=8889 \
    --no-browser --allow-root
\end{Verbatim}

La configuration STIX-Shifter connecte Kestrel aux indices Wazuh-Indexer.

\begin{Verbatim}[frame=single, numbers=left, numbersep=5pt, fontsize=\footnotesize]
# Configuration datasource Wazuh-Indexer dans Kestrel
from kestrel.session import Session

session = Session()
session.config['datasources'] = {
    'wazuh-indexer': {
        'type': 'stixshifter',
        'connection': {
            'host': '192.168.1.51',
            'port': 9200,
            'selfSignedCert': True,
            'indices': 'wazuh-alerts-*'
        },
        'connector': {
            'module': 'elastic_ecs'
        }
    }
}
\end{Verbatim}

\subsubsection{Notebook Kestrel 1 : Détection Mouvement Latéral RDP}

Ce notebook Kestrel effectue une analyse complète des connexions RDP avec statistiques et détection d'anomalies en utilisant les alertes Wazuh.

\begin{Verbatim}[frame=single, numbers=left, numbersep=5pt, fontsize=\footnotesize]
# Cellule 1: Detection connexions RDP via regles Wazuh
rdp_alerts = GET alert 
  FROM stixshifter://wazuh-indexer 
  WHERE rule_id IN ['110001', '110002', '110003']
    AND timestamp >= t'2024-11-26T00:00:00Z'

DISP rdp_alerts ATTR timestamp, agent_name, 
    rule_description, rule_level LIMIT 20

# Cellule 2: Statistiques par agent
rdp_stats = GROUP rdp_alerts BY agent_name
  COMPUTE COUNT(rdp_alerts) AS alert_count,
          MAX(rule_level) AS max_severity

DISP rdp_stats WHERE alert_count > 3

# Cellule 3: Timeline des alertes (1h bins)
rdp_timeline = GROUP rdp_alerts BY timestamp
  TIMEBIN 1h
  COMPUTE COUNT(rdp_alerts) AS alerts_per_hour

DISP rdp_timeline

# Cellule 4: Detection comptes admin cibles
admin_rdp = FILTER rdp_alerts 
  WHERE rule_id = '110002'

DISP admin_rdp ATTR timestamp, agent_name, 
    data_win_eventdata_targetUserName
\end{Verbatim}

\subsubsection{Notebook Kestrel 2 : Corrélation Pass-the-Hash}

Ce notebook corrèle le credential dumping avec les authentifications NTLM suspectes pour détecter Pass-the-Hash en utilisant les alertes Wazuh.

\begin{Verbatim}[frame=single, numbers=left, numbersep=5pt, fontsize=\footnotesize]
# Cellule 1: Detection credential dumping
lsass_alerts = GET alert 
  FROM stixshifter://wazuh-indexer 
  WHERE rule_id = '110034'
    AND timestamp >= t'2024-11-26T00:00:00Z'

DISP lsass_alerts ATTR timestamp, agent_name, 
    rule_description

# Cellule 2: Detection NTLM suspicious
pth_alerts = GET alert 
  FROM stixshifter://wazuh-indexer 
  WHERE rule_id IN ['110030', '110033']
    AND timestamp >= t'2024-11-26T00:00:00Z'

DISP pth_alerts ATTR timestamp, agent_name, 
    data_win_eventdata_workstationName

# Cellule 3: Correlation temporelle (10 min window)
pth_attack = JOIN lsass_alerts, pth_alerts
  ON lsass_alerts.agent_name = pth_alerts.agent_name
  WITHIN 10m

DISP pth_attack ATTR 
  timestamp, 
  agent_name, 
  rule_description

# Cellule 4: Filtrer severite elevee
critical_pth = FILTER pth_attack 
  WHERE rule_level >= 12

DISP critical_pth
\end{Verbatim}

\subsubsection{Notebook Kestrel 3 : Analyse WMI Event Consumers}

Ce notebook détecte la persistance via WMI Event Consumers et corrèle avec l'exécution de processus suspects.

\begin{Verbatim}[frame=single, numbers=left, numbersep=5pt, fontsize=\footnotesize]
# Cellule 1: Detection WMI Event Consumers
wmi_consumer_alerts = GET alert 
  FROM stixshifter://wazuh-indexer 
  WHERE rule_id = '110024'
    AND timestamp >= t'2024-11-26T00:00:00Z'

DISP wmi_consumer_alerts ATTR timestamp, agent_name, 
    rule_description

# Cellule 2: Detection processus lances via WMI
wmi_exec_alerts = GET alert 
  FROM stixshifter://wazuh-indexer 
  WHERE rule_id IN ['110020', '110021']
    AND timestamp >= t'2024-11-26T00:00:00Z'

# Cellule 3: Filtrer executions PowerShell critiques
critical_wmi = FILTER wmi_exec_alerts 
  WHERE rule_id = '110021'

DISP critical_wmi ATTR timestamp, agent_name, 
    data_win_eventdata_image

# Cellule 4: Correlation Consumers + Execution
persistence_attack = JOIN wmi_consumer_alerts, critical_wmi
  ON wmi_consumer_alerts.agent_name = critical_wmi.agent_name
  WITHIN 1h

DISP persistence_attack
\end{Verbatim}

\subsubsection{Notebook Kestrel 4 : Golden/Silver Ticket Detection}

Ce notebook détecte les attaques Golden Ticket et Silver Ticket via analyse des alertes Kerberos de Wazuh.

\begin{Verbatim}[frame=single, numbers=left, numbersep=5pt, fontsize=\footnotesize]
# Cellule 1: Detection TGT sans pre-auth (Golden)
golden_ticket_alerts = GET alert 
  FROM stixshifter://wazuh-indexer 
  WHERE rule_id = '110042'
    AND timestamp >= t'2024-11-26T00:00:00Z'

DISP golden_ticket_alerts ATTR timestamp, agent_name, 
    data_win_eventdata_targetUserName

# Cellule 2: Detection TGT avec RC4 (downgrade)
rc4_alerts = GET alert 
  FROM stixshifter://wazuh-indexer 
  WHERE rule_id = '110041'
    AND timestamp >= t'2024-11-26T00:00:00Z'

# Cellule 3: Detection Kerberoasting
kerberoast_alerts = GET alert 
  FROM stixshifter://wazuh-indexer 
  WHERE rule_id = '110044'
    AND timestamp >= t'2024-11-26T00:00:00Z'

kerberoast_stats = GROUP kerberoast_alerts BY agent_name
  COMPUTE COUNT(kerberoast_alerts) AS attack_count

DISP kerberoast_stats WHERE attack_count > 0

# Cellule 4: Combiner toutes les attaques Kerberos
all_kerberos_attacks = MERGE golden_ticket_alerts, 
    rc4_alerts, kerberoast_alerts

DISP all_kerberos_attacks ATTR timestamp, agent_name, 
    rule_id, rule_description
\end{Verbatim}

\subsection{Configuration des Visualisations dans Wazuh Dashboard}

Wazuh Dashboard (basé sur OpenSearch Dashboards) permet de créer des visualisations personnalisées pour surveiller les techniques de mouvement latéral d'APT41.

\subsubsection{Accès à Wazuh Dashboard}

Wazuh Dashboard est accessible via l'URL \texttt{https://192.168.1.51} avec les credentials administrateur configurés lors de l'installation. La navigation se fait via le menu latéral gauche, section \textbf{Visualize} pour créer des graphiques et \textbf{Dashboard} pour les assembler.

\subsubsection{Configuration Index Pattern}

Avant de créer des visualisations, configurez l'index pattern dans \textbf{Stack Management} → \textbf{Index Patterns} :

\begin{itemize}
    \item Index pattern : \texttt{wazuh-alerts-*}
    \item Time field : \texttt{timestamp}
    \item Refresh fields : Cliquer pour indexer tous les champs disponibles
\end{itemize}

Cette configuration permet à Wazuh Dashboard d'accéder à toutes les alertes générées par les règles personnalisées APT41.

Le tableau~\ref{tab:wazuh_dashboard_viz} présente les visualisations recommandées pour le monitoring APT41.

\begin{table}[htbp]
\centering
\caption{Visualisations Wazuh Dashboard pour Détection APT41}
\label{tab:wazuh_dashboard_viz}
\begin{tabular}{|L{3.5cm}|C{2.5cm}|L{7.5cm}|}
\hline
\textbf{Visualisation} & \textbf{Type} & \textbf{Configuration} \\
\hline
Alertes APT41 par Technique & Pie Chart & Agrégation : Terms sur rule.mitre.id, Top 5 \\
\hline
Timeline Alertes RDP & Line Chart & X-axis : timestamp (1h interval), Y-axis : Count, Filter : rule.id 110001-110005 \\
\hline
Top Agents Ciblés & Bar Chart & Y-axis : agent.name (Top 10), X-axis : Count, Filter : rule.id 110001-110055 \\
\hline
Sévérité Alertes & Gauge & Metric : Max rule.level, Color ranges : 0-10 green, 10-12 yellow, 12+ red \\
\hline
Heatmap Horaire & Heat Map & X-axis : timestamp (hour of day), Y-axis : rule.mitre.technique, Metric : Count \\
\hline
\end{tabular}
\end{table}

Ces visualisations permettent un monitoring en temps réel des activités malveillantes d'APT41 détectées par les règles Wazuh personnalisées.
