% =======================
% PREAMBULE LATEX COMPLET - AVEC ARIAL 12PT
% =======================
\documentclass[12pt,a4paper]{article}

% ---- Encodage et langue ----
\usepackage[T1]{fontenc}
\usepackage[utf8]{inputenc}
\usepackage[french]{babel}
\usepackage{textcomp}
\usepackage{csquotes}

% ---- POLICE ARIAL 12PT ----
% Utilisation de helvet (proche d'Arial) comme police par défaut
\usepackage{helvet}
\renewcommand{\familydefault}{\sfdefault}

% ---- Mise en page ----
\usepackage{geometry}
\geometry{margin=2.5cm}
\usepackage{setspace}
\onehalfspacing

% ---- Tableaux et math ----
\usepackage{array}
\usepackage{tabularx}
\usepackage{multirow}
\usepackage{booktabs}
\usepackage{amsmath, amssymb}

% ---- Graphiques et figures ----
\usepackage{graphicx}
\usepackage{float}
\usepackage{caption}
\usepackage{subcaption}

% ---- Hyperliens ----
\usepackage{hyperref}
\hypersetup{
  colorlinks=true,
  linkcolor=blue,
  citecolor=blue,
  urlcolor=blue
}

% ---- Symboles, icônes et checkmarks ----
\usepackage{pifont}
\usepackage{fontawesome}
\providecommand{\xmark}{\ding{55}}
\providecommand{\checkmark}{\ding{51}}

% ---- Code & JSON ----
\usepackage{listings}
\usepackage{xcolor}
\lstdefinelanguage{json}{
  basicstyle=\ttfamily\small,
  numbers=left,
  numberstyle=\tiny\color{gray},
  stepnumber=1,
  numbersep=6pt,
  showstringspaces=false,
  breaklines=true,
  frame=single,
  backgroundcolor=\color{gray!5},
  stringstyle=\color{green!40!black},
  keywordstyle=\color{blue},
  commentstyle=\color{gray},
  literate=
   *{0}{{{\color{black}0}}}{1}
    {1}{{{\color{black}1}}}{1}
    {2}{{{\color{black}2}}}{1}
    {3}{{{\color{black}3}}}{1}
    {4}{{{\color{black}4}}}{1}
    {5}{{{\color{black}5}}}{1}
    {6}{{{\color{black}6}}}{1}
    {7}{{{\color{black}7}}}{1}
    {8}{{{\color{black}8}}}{1}
    {9}{{{\color{black}9}}}{1}
}

% ---- Bibliographie ----
\usepackage[backend=biber,style=ieee]{biblatex}
\addbibresource{bib.bib}
\addbibresource{ietf.bib}

% ---- Autres ----
\usepackage{color}
\usepackage{fancyhdr}

% ---- Page de titre personnalisée ----
\usepackage{tikz}

% =======================
% INFORMATIONS DU DOCUMENT
% =======================
\newcommand{\documenttitle}{Détection et Réponse Automatisée aux Techniques de Mouvement Latéral de l'APT41 : Évaluation d'un SIEM Open Source avec Intégration d'Intelligence Artificielle}

\newcommand{\documentauthors}{%
    Anass Kamouni \\
    Brahim Baiteche \\
    Kouadio Bakary Ouattara \\
    Sabrine Ezzemrani
}

\newcommand{\documentinstitution}{Université de Sherbrooke}
\newcommand{\documentprogram}{Maîtrise en Cybersécurité}
\newcommand{\documentcourse}{INF808}
\newcommand{\documentsession}{Automne 2025}

% ---- Redéfinir \maketitle pour inclure le logo ----
\renewcommand{\maketitle}{
  \begin{titlepage}
    \centering
    
    % Logo de l'université en haut
    \vspace*{1cm}
    \IfFileExists{logo-usherbrooke.png}{%
      \includegraphics[width=0.3\textwidth]{logo-usherbrooke.png}
    }{%
      % Si le logo n'existe pas, afficher un emplacement
      \begin{tikzpicture}
        \draw[thick] (0,0) rectangle (4,2);
        \node at (2,1) {\small LOGO UNIVERSITÉ};
      \end{tikzpicture}
    }
    
    \vspace{0.5cm}
    
    % Institution
    {\Large \textbf{\documentinstitution} \par}
    \vspace{0.1cm}
    {\large \documentprogram \par}
    \vspace{0.1cm}
    {\large \documentcourse \par}
    
    \vspace{2cm}
    
    % Ligne horizontale supérieure
    \noindent\rule{\textwidth}{1pt}
    
    \vspace{0.5cm}
    
    % Titre
    {\LARGE \textbf{\documenttitle} \par}
    
    \vspace{0.25cm}
    
    % Ligne horizontale inférieure
    \noindent\rule{\textwidth}{1pt}
    
    \vspace{2cm}
    
    % Auteurs
    {\large
    \begin{tabular}[t]{c}
      \documentauthors
    \end{tabular}
    \par}
    
    \vfill
    
    % Session et date
    {\large \documentsession \par}
    \vspace{0.3cm}
    %{\large \today \par}
    
  \end{titlepage}
}