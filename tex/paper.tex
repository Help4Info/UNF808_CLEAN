\section{Résumé Exécutif}

\textbf{Projet :} Détection et réponse automatisée aux techniques de mouvement latéral d'APT41

\textbf{Équipe :} 4 personnes (évaluation individuelle)

\textbf{Durée :} 12-14 semaines

\textbf{Objectif :} Améliorer la détection des techniques de mouvement latéral du groupe APT41 de +15\% en combinant un SIEM open-source (Wazuh) avec des modèles d'intelligence artificielle générative pour l'analyse et la réponse automatisée aux incidents.

\textbf{5 Techniques ciblées :}
\begin{itemize}
    \item T1021.001 - Remote Desktop Protocol (RDP)
    \item T1021.002 - SMB/Windows Admin Shares (PsExec)
    \item T1047 - Windows Management Instrumentation (WMI)
    \item T1550.002 - Pass-the-Hash (PtH)
    \item T1550.003 - Pass-the-Ticket (PtT)
\end{itemize}

\textbf{Outils principaux :} Wazuh (SIEM), Caldera (Red Team), Velociraptor (DFIR), OpenAI/Claude/Gemini (IA), JIRA (ticketing), Shuffle (SOAR)

\textbf{Livrables finaux :}
\begin{itemize}
    \item Rapport LaTeX 30-50 pages
    \item 16+ règles Wazuh personnalisées
    \item Pipeline SOAR complet et opérationnel
    \item Dataset annoté de 500+ événements
    \item Présentation 15-20 minutes
\end{itemize}

\section{Composition de l'Équipe}

Le projet se fait en GROUPE, mais l'évaluation est \textbf{PERSONNELLE}. Les contributions seront tracées via GitHub (commits individuels).

\begin{table}[H]
\centering
\begin{tabular}{|l|l|p{6cm}|c|}
\hline
\textbf{Membre} \& \textbf{Rôle} \& \textbf{Responsabilités} \& \textbf{\% Travail} \\
\hline
Anass Kamouni \& Red Team \& Caldera, simulations APT41 \& 25\% \\
\hline
Brahim  Baiteche \& Blue Team \& Wazuh SIEM, règles détection \& 25\% \\
\hline
 Kouadio Bakary  Ouattara \& Purple Team \& Velociraptor, IA, SOAR \& 25\% \\
\hline
Sabrine  Ezzemrani \& Documentation \& État de l'art, rapport LaTeX \& 25\% \\
\hline
\end{tabular}
\caption{Répartition du travail entre les membres}
\label{tab:equipe}
\end{table}

\section{Problématique}

\subsection{Question de Recherche}

\emph{``Comment améliorer la détection des techniques de mouvement latéral spécifiques au groupe APT41 dans un environnement d'entreprise en combinant les capacités d'un SIEM open-source (Wazuh) avec des modèles d'intelligence artificielle pour l'analyse et la réponse automatisée aux incidents ?''}

\subsection{Contexte}

Le groupe APT41 (alias Barium, Winnti, Wicked Panda) est un acteur de menace persistante avancée d'origine chinoise actif depuis 2012. Ce groupe se distingue par :

\begin{itemize}
    \item \textbf{Campagnes hybrides :} Combinaison d'espionnage étatique et de cybercriminalité financière
    \item \textbf{Techniques sophistiquées :} Utilisation extensive de mouvements latéraux dans les infrastructures Active Directory
    \item \textbf{Objectifs variés :} Vol de propriété intellectuelle, compromission de supply chain, ransomware
\end{itemize}

Les techniques de mouvement latéral d'APT41 sont particulièrement difficiles à détecter car elles abusent de protocoles légitimes (RDP, WMI, SMB) et d'outils natifs Windows.

\subsection{Objectifs Mesurables}

\begin{enumerate}
    \item \textbf{Amélioration de la détection :} +15\% de taux de détection par rapport aux règles Wazuh par défaut
    \item \textbf{Réduction des faux positifs :} -20\% grâce à l'analyse contextuelle par IA
    \item \textbf{Automatisation :} Réponse aux incidents en moins de 5 minutes via pipeline SOAR
    \item \textbf{Dataset partageable :} 500+ événements annotés avec TTPs MITRE ATT\&CK
    \item \textbf{Solution 100\% open-source :} Reproductible par la communauté (sauf IA)
\end{enumerate}

\subsection{Contributions Originales}

\begin{itemize}
    \item Règles Wazuh personnalisées optimisées pour APT41 (non disponibles dans la communauté)
    \item Intégration IA générative (OpenAI/Claude/Gemini) pour enrichissement automatique d'incidents
    \item Pipeline SOAR complet : Wazuh $\rightarrow$ IA $\rightarrow$ JIRA $\rightarrow$ Email
    \item Guide de reproduction : Documentation technique complète pour réplication
    \item Dataset annoté : Logs d'attaques APT41 simulées avec labels MITRE ATT\&CK
\end{itemize}

\section{État de l'Art Préliminaire}

\subsection{Groupe APT41 et Techniques de Mouvement Latéral}

\textbf{Profil de l'acteur de menace :}
\begin{itemize}
    \item \textbf{MITRE ATT\&CK ID :} G0096
    \item \textbf{Alias :} Barium, Winnti, Wicked Panda
    \item \textbf{Origine :} Chine (état + cybercriminalité)
    \item \textbf{Actif depuis :} 2012
    \item \textbf{Secteurs ciblés :} Santé, télécommunications, jeux vidéo, finance, gouvernement
\end{itemize}

\begin{table}[H]
\centering
\begin{tabular}{|l|p{5cm}|c|}
\hline
\textbf{Technique MITRE} \& \textbf{Description} \& \textbf{Prévalence APT41} \\
\hline
T1021.001 \& Remote Desktop Protocol (RDP) \& Très élevée $\star\star\star$ \\
\hline
T1021.002 \& SMB/Admin Shares (PsExec) \& Très élevée $\star\star\star$ \\
\hline
T1047 \& Windows Management Instrumentation \& Élevée $\star\star$ \\
\hline
T1550.002 \& Pass-the-Hash (PtH) \& Élevée $\star\star$ \\
\hline
T1550.003 \& Pass-the-Ticket (PtT) - Kerberos \& Moyenne $\star$ \\
\hline
\end{tabular}
\caption{Techniques de mouvement latéral prioritaires pour APT41}
\label{tab:techniques-apt41}
\end{table}

\textbf{Références clés :}
\begin{itemize}
    \item MITRE ATT\&CK - APT41 Group Profile (G0096) \cite{mitre-apt41}
    \item Mandiant, ``APT41: A Dual Espionage and Cyber Crime Operation'', 2019 \cite{mandiant-apt41}
    \item FireEye, ``Double Dragon: APT41'', 2019 \cite{fireeye-apt41}
    \item CISA Alert (AA21-200A), ``Chinese State-Sponsored Cyber Operations'' \cite{cisa-apt41}
\end{itemize}

\subsection{SIEM Open-Source et Wazuh}

Wazuh est un SIEM open-source basé sur OSSEC, offrant :
\begin{itemize}
    \item Détection d'intrusions basée sur règles (XML)
    \item Collecte centralisée de logs (agents Windows/Linux)
    \item Intégration MITRE ATT\&CK native
    \item API RESTful pour automatisation
    \item Dashboard Kibana/OpenSearch
\end{itemize}

\textbf{Avantages pour ce projet :}
\begin{itemize}
    \item[✓] 100\% open-source (objectif du projet)
    \item[✓] Support natif Sysmon (détection bas-niveau Windows)
    \item[✓] Règles personnalisables (XML)
    \item[✓] Communauté active (règles Sigma converties)
\end{itemize}

\textbf{Limitations identifiées :}
\begin{itemize}
    \item[] Règles génériques (peu spécifiques à APT41)
    \item[] Faux positifs élevés (environnements AD légitimes)
    \item[] Pas d'analyse contextuelle (IA requise)
\end{itemize}

\subsection{Intelligence Artificielle pour Détection d'Intrusions}

\begin{table}[H]
\centering
\small
\begin{tabular}{|p{15cm}|p{15cm}|p{10cm}|}
\hline
\textbf{Approche} \& \textbf{Avantages} \& \textbf{Limitations} \\
\hline
Règles expertes \& Précision élevée, explicabilité \& Maintenabilité, évolution \\
\hline
ML non supervisé \& Détection d'anomalies \& Faux positifs élevés \\
\hline
ML supervisé \& Bonne précision \& Nécessite dataset annoté \\
\hline
Deep Learning \& Séquences complexes \& Coût, interprétabilité \\
\hline
IA générative \& Analyse contextuelle \& Coût API, latence \\
\hline
\end{tabular}
\caption{Comparaison des approches d'IA pour détection d'intrusions}
\label{tab:ia-approches}
\end{table}

\textbf{Notre approche hybride :}
\begin{enumerate}
    \item \textbf{Couche 1 :} Règles Wazuh expertes (détection initiale rapide)
    \item \textbf{Couche 2 :} IA générative (enrichissement contextuel des incidents)
    \item \textbf{Couche 3 :} Automatisation SOAR (réponse orchestrée)
\end{enumerate}

\subsection{SOAR et Automatisation}

SOAR (Security Orchestration, Automation and Response) permet :
\begin{itemize}
    \item Orchestration multi-outils (SIEM, DFIR, ticketing)
    \item Automatisation des playbooks de réponse
    \item Réduction du temps de réponse (MTTR)
\end{itemize}

\textbf{Pipeline proposé :}
\begin{center}
Wazuh Alert $\rightarrow$ API Webhook $\rightarrow$ IA Enrichissement $\rightarrow$ JIRA Incident $\rightarrow$ Email SOC $\rightarrow$ Velociraptor Hunt
\end{center}

\textbf{Temps de réponse ciblé :}
\begin{itemize}
    \item Détection Wazuh : $<$10 secondes
    \item Enrichissement IA : $<$30 secondes
    \item Création incident JIRA : $<$60 secondes
    \item Hunting Velociraptor : $<$5 minutes
    \item \textbf{Total MTTR : $<$5 minutes}
\end{itemize}

\subsection{Digital Forensics and Incident Response (DFIR)}

Velociraptor est un framework DFIR open-source permettant :
\begin{itemize}
    \item Collecte d'artefacts à grande échelle (VQL queries)
    \item Hunting proactif (recherche d'IOCs)
    \item Timeline reconstruction
    \item Réponse automatisée (isolation, collecte mémoire)
\end{itemize}

\textbf{Intégration dans le projet :}
\begin{itemize}
    \item Déploiement sur toutes les machines Windows
    \item Hunting automatique post-détection Wazuh
    \item Collecte d'artefacts pour analyse forensique
    \item Validation des détections (ground truth)
\end{itemize}

\section{Formalisations des Attaques}

\subsection{Mapping MITRE ATT\&CK}

\begin{table}[H]
\centering
\tiny
\begin{tabular}{|l|p{3cm}|l|c|c|c|}
\hline
\textbf{Tactic} \& \textbf{Technique} \& \textbf{ID} \& \textbf{Priorité} \& \textbf{Wazuh} \& \textbf{Velo} \\
\hline
Lateral Mvt \& Remote Services: RDP \& T1021.001 \& $\star\star\star$ \& ✓ \& ✓ \\
\hline
Lateral Mvt \& Remote Services: SMB \& T1021.002 \& $\star\star\star$ \& ✓ \& ✓ \\
\hline
Execution \& WMI \& T1047 \& $\star\star$ \& ✓ \& ✓ \\
\hline
Credential Access \& LSASS Dump \& T1003.001 \& $\star\star\star$ \& ✓ \& ✓ \\
\hline
Lateral Mvt \& Pass-the-Hash \& T1550.002 \& $\star\star$ \& ✓ \& ✓ \\
\hline
Lateral Mvt \& Pass-the-Ticket \& T1550.003 \& $\star$ \& ✓ \& ✓ \\
\hline
\end{tabular}
\caption{Matrice de couverture des techniques APT41}
\label{tab:mitre-coverage}
\end{table}

\subsection{Modèle STIX 2.1}

Exemple de formalisation STIX pour T1021.002 (PsExec Lateral Movement) :

\begin{verbatim}
{
  "type": "attack-pattern",
  "spec_version": "2.1",
  "id": "attack-pattern--54a649ff-439a-41a4-9856-8d144a2551ba",
  "name": "Remote Services: SMB/Windows Admin Shares",
  "description": "APT41 utilise PsExec via SMB",
  "kill_chain_phases": [{
    "kill_chain_name": "mitre-attack",
    "phase_name": "lateral-movement"
  }],
  "external_references": [{
    "source_name": "mitre-attack",
    "external_id": "T1021.002"
  }]
}
\end{verbatim}

\subsection{Cyber Kill Chain}

\begin{table}[H]
\centering
\small
\begin{tabular}{|l|p{4cm}|c|c|}
\hline
\textbf{Phase} \& \textbf{Actions APT41} \& \textbf{Wazuh} \& \textbf{Velo} \\
\hline
1. Reconnaissance \& Scan réseau \&  \& ✓ \\
\hline
2. Weaponization \& Préparation exploits \&  \&  \\
\hline
3. Delivery \& Phishing, supply chain \&  \&  \\
\hline
4. Exploitation \& Exploitation vulnérabilités \& ⚠ \& ✓ \\
\hline
5. Installation \& Persistence \& ✓ \& ✓ \\
\hline
6. Command \& Control \& Beacon C2 \& ✓ \& ✓ \\
\hline
7. \textbf{Actions on Objectives} \& \textbf{Lateral Movement} \& \textbf{✓} \& \textbf{✓} \\
\hline
\end{tabular}
\caption{Cyber Kill Chain et couverture de détection (Focus : Phase 7)}
\label{tab:kill-chain}
\end{table}

\section{Architecture Technique}

\subsection{Infrastructure de Laboratoire}

Configuration réseau isolé (5 VMs) :

\begin{itemize}
    \item \textbf{DC01} (Windows Server 2022) - 192.168.100.10 : Active Directory + DNS
    \item \textbf{WIN-CLI01} (Windows 10 Pro) - 192.168.100.20 : Client AD + cible
    \item \textbf{WIN-CLI02} (Windows 10 Pro) - 192.168.100.21 : Client AD + cible
    \item \textbf{WAZUH-MGR} (Ubuntu 22.04) - 192.168.100.5 : Wazuh Manager + Velociraptor
    \item \textbf{KALI-ATTACKER} (Kali Linux) - 192.168.100.50 : Caldera + Outils Red Team
\end{itemize}

\begin{table}[H]
\centering
\small
\begin{tabular}{|l|l|c|c|c|p{4cm}|}
\hline
\textbf{VM} \& \textbf{OS} \& \textbf{vCPU} \& \textbf{RAM} \& \textbf{Disque} \& \textbf{Rôle} \\
\hline
DC01 \& Win Server 2022 \& 2 \& 4 GB \& 60 GB \& Active Directory \\
\hline
WIN-CLI01 \& Windows 10 Pro \& 2 \& 4 GB \& 40 GB \& Client + Cible \\
\hline
WIN-CLI02 \& Windows 10 Pro \& 2 \& 4 GB \& 40 GB \& Client + Cible \\
\hline
WAZUH-MGR \& Ubuntu 22.04 \& 4 \& 8 GB \& 80 GB \& SIEM + DFIR \\
\hline
KALI-ATK \& Kali Linux \& 2 \& 4 GB \& 60 GB \& Red Team \\
\hline
\textbf{Total} \&  \& \textbf{12} \& \textbf{24 GB} \& \textbf{280 GB} \&  \\
\hline
\end{tabular}
\caption{Spécifications techniques de l'infrastructure}
\label{tab:infra-specs}
\end{table}

\subsection{Stack Technologique}

\begin{table}[H]
\centering
\small
\begin{tabular}{|l|l|l|l|p{3cm}|}
\hline
\textbf{Catégorie} \& \textbf{Outil} \& \textbf{Version} \& \textbf{Licence} \& \textbf{Rôle} \\
\hline
SIEM \& Wazuh \& 4.8+ \& GPL v2 \& Détection \\
\hline
DFIR \& Velociraptor \& 0.7+ \& AGPL v3 \& Forensique \\
\hline
Red Team \& Caldera \& 5.0+ \& Apache 2.0 \& Simulation \\
\hline
Logging \& Sysmon \& 15+ \& Microsoft \& Télémétrie \\
\hline
IA \& OpenAI/Claude \& API \& Commercial \& Enrichissement \\
\hline
SOAR \& Shuffle/n8n \& Latest \& AGPL v3 \& Automatisation \\
\hline
Ticketing \& JIRA \& Cloud \& Commercial \& Incidents \\
\hline
\end{tabular}
\caption{Stack technologique du projet}
\label{tab:stack}
\end{table}

\section{Répartition du Travail}

\subsection{Membre 1 - Red Team \& Simulation (25\%)}

\textbf{Responsabilités principales :}
\begin{itemize}
    \item Installation et configuration Caldera
    \item Création profils adversaires APT41 (5 techniques minimum)
    \item Exécution 20+ simulations d'attaques variées
    \item Génération dataset annoté (500+ événements)
    \item Documentation techniques d'attaque
\end{itemize}

\textbf{Livrables :}
\begin{itemize}
    \item Serveur Caldera opérationnel (Semaine 4)
    \item 5 profils adversaires YAML (Semaine 6)
    \item Dataset logs annotés (Semaine 9)
    \item Journal de simulation (Markdown/CSV)
    \item Contribution état de l'art (section Red Team)
\end{itemize}

\textbf{Justification implication personnelle :}
\begin{itemize}
    \item Commits GitHub : Configuration Caldera + profils YAML
    \item Dataset : Fichiers CSV/JSON avec timestamps et hash
    \item Documentation : Guides techniques + screenshots
\end{itemize}

\subsection{Membre 2 - Blue Team \& Détection SIEM (25\%)}

\textbf{Responsabilités principales :}
\begin{itemize}
    \item Déploiement infrastructure Wazuh (manager + agents)
    \item Développement 16+ règles de détection personnalisées
    \item Configuration dashboards et alertes
    \item Validation détection (vs simulations Membre 1)
    \item Optimisation réduction faux positifs
\end{itemize}

\textbf{Livrables :}
\begin{itemize}
    \item Wazuh déployé sur 4 agents (Semaine 5)
    \item 16 règles XML custom (Semaine 9)
    \item Dashboard Kibana APT41 (Semaine 10)
    \item Rapport métriques (taux détection, FP/FN)
    \item Contribution état de l'art (section SIEM)
\end{itemize}

\subsection{Membre 3 - Purple Team, IA \& Automatisation (25\%)}

\textbf{Responsabilités principales :}
\begin{itemize}
    \item Installation Velociraptor + création VQL queries
    \item Intégration IA (OpenAI/Claude/Gemini) avec Wazuh
    \item Développement pipeline SOAR (Wazuh $\rightarrow$ IA $\rightarrow$ JIRA)
    \item Automatisation emails + enrichissement incidents
    \item Tests end-to-end du pipeline
\end{itemize}

\textbf{Livrables :}
\begin{itemize}
    \item Velociraptor déployé + 5 hunts VQL (Semaine 9)
    \item Script Python intégration IA (Semaine 10)
    \item Pipeline SOAR fonctionnel (Semaine 11)
    \item Démo vidéo workflow complet (Semaine 12)
    \item Contribution état de l'art (section IA + SOAR)
\end{itemize}

\subsection{Membre 4 - Documentation \& Rapport (25\%)}

\textbf{Responsabilités principales :}
\begin{itemize}
    \item Recherche état de l'art (30+ références)
    \item Création diagrammes STIX + MITRE ATT\&CK
    \item Rédaction rapport LaTeX (30-50 pages)
    \item Maintien README.md + documentation technique
    \item Préparation présentation finale (slides)
\end{itemize}

\textbf{Livrables :}
\begin{itemize}
    \item État de l'art (10 pages) avec 30+ références (Semaine 3)
    \item Diagrammes STIX/ATT\&CK (Semaine 4)
    \item Sections méthodologie + expérimentation (Semaines 9-11)
    \item Rapport final PDF (Semaine 13)
    \item Présentation PowerPoint/Beamer (Semaine 13)
\end{itemize}

\section{Timeline Détaillée (12 Semaines)}

\subsection{Semaines 1-2 : Recherche \& Planification}

\textbf{Objectifs :}
\begin{itemize}
    \item Validation projet avec professeur
    \item Recherche bibliographique initiale
    \item Setup environnement GitHub
\end{itemize}

\textbf{Jalons :}
\begin{itemize}
    \item[✓] Projet validé par professeur
    \item[✓] Repo GitHub actif (1er commit)
    \item[✓] Rôles clairement définis
\end{itemize}

\subsection{Semaines 3-5 : Setup Infrastructure}

\textbf{Objectifs :}
\begin{itemize}
    \item Déployer lab 5 VMs
    \item Installer Wazuh + Caldera
    \item Configurer Active Directory
\end{itemize}

\textbf{Jalons :}
\begin{itemize}
    \item[✓] 5 VMs opérationnelles
    \item[✓] Wazuh collecte logs de 3 agents
    \item[✓] Caldera contrôle 2 agents
\end{itemize}

\subsection{Semaines 6-7 : Développement Capacités}

\textbf{Objectifs :}
\begin{itemize}
    \item Créer profils adversaires APT41
    \item Développer règles Wazuh personnalisées
    \item Première simulation test
\end{itemize}

\textbf{Jalons :}
\begin{itemize}
    \item[✓] 5 profils Caldera créés
    \item[✓] 10+ règles Wazuh déployées
    \item[✓] Première détection réussie
\end{itemize}

\subsection{Semaines 8-9 : Simulations \& Optimisation}

\textbf{Objectifs :}
\begin{itemize}
    \item Exécuter 20+ simulations variées
    \item Optimiser règles (réduire faux positifs)
    \item Annoter dataset complet
\end{itemize}

\textbf{Jalons :}
\begin{itemize}
    \item[✓] 20+ simulations complètes
    \item[✓] Dataset 500+ événements
    \item[✓] Taux détection $\geq$85\%
    \item[✓] Velociraptor déployé
\end{itemize}

\subsection{Semaines 10-11 : Intégration IA \& SOAR}

\textbf{Objectifs :}
\begin{itemize}
    \item Intégrer IA générative avec Wazuh
    \item Créer pipeline automatisé complet
    \item Tester end-to-end
\end{itemize}

\textbf{Jalons :}
\begin{itemize}
    \item[✓] Pipeline SOAR fonctionnel
    \item[✓] IA enrichit 100\% des incidents
    \item[✓] MTTR $<$5 minutes validé
    \item[✓] Démo vidéo enregistrée
\end{itemize}

\subsection{Semaines 12-13 : Rapport Final \& Présentation}

\textbf{Objectifs :}
\begin{itemize}
    \item Finaliser rapport LaTeX (30-50 pages)
    \item Préparer présentation (15-20 minutes)
    \item Derniers tests et validations
\end{itemize}

\textbf{Jalons :}
\begin{itemize}
    \item[✓] Rapport PDF finalisé
    \item[✓] Présentation prête
    \item[✓] Tous les livrables sur GitHub
    \item[✓] Projet prêt pour soutenance
\end{itemize}

\section{Premières Tâches (Semaines 1-2)}

\subsection{Tâche 1 : Configuration GitHub (TOUS) - 2h}

\textbf{Objectif :} Repository GitHub opérationnel avec structure projet

\textbf{Structure dossiers :}
\begin{verbatim}
/rapport/        (LaTeX source)
/scripts/        (Python SOAR)
/rules/          (Wazuh XML)
/caldera/        (Profils YAML)
/docs/           (Documentation MD)
/datasets/       (Logs annotés - .gitignore)
\end{verbatim}

\textbf{Critères de réussite :}
\begin{itemize}
    \item[✓] Repo accessible aux 4 membres
    \item[✓] Structure dossiers créée
    \item[✓] Template LaTeX présent
    \item[✓] Chaque membre a cloné le repo
\end{itemize}

\subsection{Tâche 2 : État de l'Art Préliminaire (MEMBRE 4) - 8h}

\textbf{Objectif :} 15 références clés identifiées et résumées

\textbf{Critères de réussite :}
\begin{itemize}
    \item[✓] 15+ références dans bib.bib
    \item[✓] Synthèse 3 pages
    \item[✓] Commit GitHub
\end{itemize}

\subsection{Tâche 3 : Diagrammes STIX/ATT\&CK (MEMBRE 4) - 4h}

\textbf{Objectif :} Formaliser 3 techniques APT41 en STIX 2.1

\textbf{Critères de réussite :}
\begin{itemize}
    \item[✓] 3 fichiers JSON STIX valides
    \item[✓] 3 diagrammes PNG/SVG
    \item[✓] Commit GitHub
\end{itemize}

\subsection{Tâche 4 : Tests Outils Préliminaires (MEMBRES 1-2-3) - 6h}

\textbf{Membre 1 (Red Team) :}
\begin{itemize}
    \item Installer Caldera sur VM test
    \item Créer 1 ability simple
    \item Déployer 1 agent Sandcat
    \item Livrable : Screenshot dashboard
\end{itemize}

\textbf{Membre 2 (Blue Team) :}
\begin{itemize}
    \item Installer Wazuh Manager
    \item Déployer 1 agent Wazuh
    \item Configurer Sysmon
    \item Livrable : Screenshot dashboard
\end{itemize}

\textbf{Membre 3 (Purple Team) :}
\begin{itemize}
    \item Tester API OpenAI/Claude/Gemini
    \item Script Python minimal
    \item Tester JIRA API
    \item Livrable : Script + screenshot
\end{itemize}

\section{Ressources et Références}

\subsection{Documentation Officielle}

\textbf{Outils principaux :}
\begin{itemize}
    \item Wazuh : \url{https://documentation.wazuh.com/}
    \item Caldera : \url{https://caldera.readthedocs.io/}
    \item Velociraptor : \url{https://docs.velociraptor.app/}
    \item MITRE ATT\&CK : \url{https://attack.mitre.org/groups/G0096/}
    \item Shuffle : \url{https://shuffler.io/docs}
    \item STIX 2.1 : \url{https://docs.oasis-open.org/cti/stix/v2.1/}
\end{itemize}

\textbf{APIs :}
\begin{itemize}
    \item OpenAI : \url{https://platform.openai.com/docs/}
    \item Anthropic Claude : \url{https://docs.anthropic.com/}
    \item Google Gemini : \url{https://ai.google.dev/docs}
    \item JIRA API : \url{https://developer.atlassian.com/cloud/jira/}
\end{itemize}

\subsection{Datasets Publics}

\begin{itemize}
    \item CICIDS2017 : \url{https://www.unb.ca/cic/datasets/ids-2017.html}
    \item NSL-KDD : \url{https://www.unb.ca/cic/datasets/nsl.html}
    \item Mordor Security Dataset : \url{https://github.com/OTRF/Security-Datasets}
    \item Atomic Red Team : \url{https://github.com/redcanaryco/atomic-red-team}
\end{itemize}



\section{Risques et Mitigation}

\subsection{Risques Techniques}

\begin{table}[H]
\centering
\small
%\begin{tabular}{|p{10cm}|c|c|p{4cm}|}
\begin{tabular}{|l|l|c|c|c|p{4cm}|}
\hline
\textbf{Risque} \& \textbf{Prob.} \& \textbf{Impact} \& \textbf{Mitigation} \\
\hline
Infra VM insuffisante \& Moy \& Élevé \& Cloud (AWS Free Tier) \\
\hline
API IA coûteuses \& Moy \& Moy \& Limiter appels, Ollama local \\
\hline
Caldera bloqué par AV \& Élevé \& Moy \& Environnement isolé \\
\hline
Logs trop volumineux \& Moy \& Moy \& Filtrage, rotation \\
\hline
Intégration SOAR \& Élevé \& Élevé \& Tests unitaires, MVP \\
\hline
\end{tabular}
\caption{Risques techniques et stratégies de mitigation}
\label{tab:risques-tech}
\end{table}

\subsection{Plan de Contingence}

\textbf{Scénario 1 : Retard infrastructure (Semaine 6)}
\begin{itemize}
    \item Utiliser VMs pré-configurées
    \item Réduire à 3 VMs (DC + 1 client + Wazuh)
\end{itemize}

\textbf{Scénario 2 : API IA trop coûteuse (Semaine 10)}
\begin{itemize}
    \item Utiliser Ollama (Llama 3.2 local gratuit)
    \item Prompts simplifiés
\end{itemize}

\textbf{Scénario 3 : Détection insuffisante ($<$70\%)}
\begin{itemize}
    \item Réduire scope à 3 techniques
    \item Focus sur T1021.001 + T1021.002 + T1003.001
\end{itemize}

\section{Livrables Finaux}

\subsection{Rapport Technique}

\begin{itemize}
    \item \textbf{Format :} PDF généré depuis LaTeX
    \item \textbf{Pages :} 30-50 pages
    \item \textbf{Structure :}
    \begin{enumerate}
        \item Introduction (Problématique) - 4 pages
        \item État de l'art - 10 pages
        \item Méthodologie - 8 pages
        \item Expérimentation - 12 pages
        \item Conclusion - 3 pages
        \item Références - 3 pages
        \item Annexes - 10 pages
    \end{enumerate}
\end{itemize}

\subsection{Code Source (GitHub)}

\textbf{Repository :} \url{https://github.com/[USERNAME]/APT41-Detection-Project}

\textbf{Contenu :}
\begin{itemize}
    \item Règles Wazuh XML (16+ fichiers)
    \item Profils Caldera YAML (5+ fichiers)
    \item Scripts Python SOAR (3+ fichiers)
    \item VQL Hunts Velociraptor (5+ fichiers)
    \item Diagrammes STIX/ATT\&CK
    \item Documentation complète
    \item Dataset annoté (CSV)
\end{itemize}

\subsection{Présentation}

\begin{itemize}
    \item \textbf{Format :} PowerPoint / Beamer (PDF)
    \item \textbf{Durée :} 15-20 minutes
    \item \textbf{Slides :} 20-25 slides
    \item \textbf{Démo vidéo :} 5-10 minutes (optionnel mais recommandé)
\end{itemize}

\section{Contact et Validation}

\subsection{Réunion de Validation (Semaine 2 - OBLIGATOIRE)}

\textbf{À préparer pour la réunion :}
\begin{itemize}
    \item[✓] Ce document (plan de projet)
    \item[✓] Diagrammes STIX/ATT\&CK
    \item[✓] État de l'art préliminaire
    \item[✓] Architecture infrastructure
    \item[✓] Repository GitHub
\end{itemize}

\textbf{Questions à poser au professeur :}
\begin{enumerate}
    \item Scope acceptable ? (5 techniques vs. 3 minimum ?)
    \item Intégration IA considérée comme contribution originale ?
    \item Dataset annoté : format recommandé ?
    \item Rapport : 30-50 pages inclut annexes ?
    \item Présentation : démo live ou vidéo ?
\end{enumerate}

\subsection{Suivi Bihebdomadaire/Mensuel}

\textbf{Dates suggérées (à confirmer) :}
\begin{itemize}
    \item Semaine 2 : Validation initiale
    \item Semaine 5 : Revue infrastructure
    \item Semaine 8 : Revue simulations/détection
    \item Semaine 11 : Revue intégration IA/SOAR
    \item Semaine 13 : Présentation finale
\end{itemize}

\section{Conclusion}

Ce plan de projet présente une approche structurée et réaliste pour améliorer la détection des techniques de mouvement latéral d'APT41 en combinant un SIEM open-source (Wazuh) avec l'intelligence artificielle générative.

\textbf{Points clés :}
\begin{itemize}
    \item Objectifs mesurables et atteignables
    \item Répartition équitable du travail (25\% par membre)
    \item Timeline réaliste sur 12-14 semaines
    \item Solution 100\% open-source (sauf IA)
    \item Contributions originales clairement identifiées
    \item Plan de contingence pour les risques
\end{itemize}

\textbf{Prochaine étape :} Validation avec le professeur en Semaine 2.

\vspace{1cm}

\noindent
%\textbf{Version :} 1.0 \\
%\textbf{Date de création :} 2025-11-08 \\
%\textbf{Statut :} En attente de validation professeur
