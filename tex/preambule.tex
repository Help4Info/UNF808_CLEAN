% ===================================================================
% INFORMATIONS DU DOCUMENT
% ===================================================================

\title{État de l'Art : Détection des Techniques de Mouvement Latéral du Groupe APT41}

\author{
  \textbf{Étudiant 1 : Prénom NOM} \\
  Matricule : XXXXXXX \\[0.2cm]
  \textbf{Étudiant 2 : Prénom NOM} \\
  Matricule : XXXXXXX \\[0.2cm]
  \textbf{Étudiant 3 : Prénom NOM} \\
  Matricule : XXXXXXX \\[0.2cm]
  \textbf{Étudiant 4 : Prénom NOM} \\
  Matricule : XXXXXXX
}

\date{22 novembre 2024}

% Informations supplémentaires pour la page de garde
\newcommand{\institution}{UNIVERSITÉ DE SHERBROOKE (UDS)}
\newcommand{\departement}{}
\newcommand{\programme}{Maîtrise en Cybersécurité}
\newcommand{\cours}{INF808 - Réaction aux attaques et analyses des attaques}
\newcommand{\session}{Automne 2025}
\newcommand{\professeur}{Professeur : Daniel Migault}

% ===================================================================
% PACKAGES POUR LISTINGS ET COLORATION SYNTAXIQUE
% ===================================================================

% Package pour listings avec support YAML, PowerShell, Python
\usepackage{listings}
\usepackage{xcolor}
\usepackage{tikz}              % Diagrammes
\usepackage{pgfgantt}          % Gantt charts
\usepackage{booktabs}          % Tables professionnelles
\usepackage{float}             % Position figures [H]
% Configuration listings
\lstset{
    basicstyle=\ttfamily\small,
    breaklines=true,
    frame=single,
    numbers=left,
    numberstyle=\tiny,
    captionpos=b,
    backgroundcolor=\color{gray!10}
}

% Définition des couleurs pour YAML
\definecolor{yamlkey}{RGB}{0,0,255}
\definecolor{yamlstring}{RGB}{0,128,0}
\definecolor{yamlcomment}{RGB}{128,128,128}
\definecolor{yamlvalue}{RGB}{163,21,21}
\definecolor{yamlbackground}{RGB}{248,248,248}

% Configuration du langage YAML pour listings
\lstdefinelanguage{yaml}{
  keywords={id, name, description, tactic, technique, attack_id, platforms, windows, psh, command, cleanup, requirements, access, min_clearance, parsers, source},
  keywordstyle=\color{yamlkey}\bfseries,
  string=[s]{"}{"},
  stringstyle=\color{yamlstring},
  comment=[l]{\#},
  commentstyle=\color{yamlcomment}\itshape,
  morecomment=[s]{/*}{*/},
  basicstyle=\ttfamily\small,
  breaklines=true,
  frame=single,
  numbers=left,
  numberstyle=\tiny\color{gray},
  showstringspaces=false,
  backgroundcolor=\color{yamlbackground}
}

% Configuration du langage PowerShell pour listings
\lstdefinelanguage{PowerShell}{
  keywords={function, if, else, foreach, while, return, param, begin, process, end, try, catch, finally},
  keywordstyle=\color{blue}\bfseries,
  string=[b]",
  stringstyle=\color{red},
  comment=[l]{\#},
  commentstyle=\color{gray}\itshape,
  morecomment=[s]{<\#}{\#>},
  basicstyle=\ttfamily\small,
  breaklines=true,
  showstringspaces=false
}

% Configuration globale des listings
\lstset{
  basicstyle=\ttfamily\footnotesize,
  numbers=left,
  numberstyle=\tiny\color{gray},
  stepnumber=1,
  numbersep=5pt,
  backgroundcolor=\color{yamlbackground},
  showspaces=false,
  showstringspaces=false,
  showtabs=false,
  frame=single,
  rulecolor=\color{black},
  tabsize=2,
  captionpos=b,
  breaklines=true,
  breakatwhitespace=false,
  breakindent=0pt,
  postbreak=\mbox{\textcolor{gray}{$\hookrightarrow$}\space},
  escapeinside={\%*}{*)},
  xleftmargin=2em,
  framexleftmargin=1.5em,
  language=yaml
}

% Style pour code inline
\newcommand{\code}[1]{\texttt{#1}}
\newcommand{\file}[1]{\texttt{#1}}
\newcommand{\cmd}[1]{\texttt{#1}}